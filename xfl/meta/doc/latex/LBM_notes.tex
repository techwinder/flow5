\documentclass[12pt]{article}
\usepackage[a4paper, portrait, margin=1in]{geometry}
\usepackage{mathtools}
\usepackage[export]{adjustbox}
\usepackage{color}
\usepackage{amsmath}
\usepackage{hyperref}
\usepackage{xfrac}

\graphicspath{{./figures/}}
\renewcommand{\familydefault}{\sfdefault}

 
\begin{document}

\title{Lattice Boltzmann Method - Implementation notes}
\author{techwinder}

\maketitle


\setlength{\parindent}{0pt}
\setlength{\parskip}{1em}

\section{Introduction}
This document is a summary of the notes taken during the implementation in foil5 of an LBM 2d flow solver for airfoils. 
\section{Lattice description}
foil5 implements the d2q9 operator, with the convention illustrated in figure \ref{naca12top}.

\begin{figure} 
\centering
\includegraphics[width=10cm]{d2q9.png}
\caption{Directions of distributions}
\label{naca12top}
\end{figure}

To simplify the imposition of boundary conditions, the velocity vectors have unit lengths 1 for the directions 1, 2, 3 and 4, and length $\sqrt{2}$ along the diagonal directions:
\begin{equation}
\label{Directions}
\begin{split}
c_0 &= (0,0) \\
c_1 &= (1,0) \\
c_2 &= (0,1) \\
c_3 &= (-1,0) \\
c_4 &= (0,-1) \\
c_5 &= (1,1) \\
c_6 &= (-1,1) \\
c_7 &= (-1,-1) \\
c_8 &= (1,-1) \\
\end{split}
\end{equation}

The weight factors of the distributions are:
\begin{equation}
\begin{split}
\label{Weight}
w_0 &= \frac{4}{9}\\
w_1 = w_2 = w_3 = w_4 &= \frac{1}{9}\\
w_5 = w_6 = w_7 = w_8 &= \frac{1}{36}
\end{split}
\end{equation}


The equilibrium distributions are:
\begin{equation}
f_i^{eq} =  w_i \, \rho \, \big( 1 + 3 \, c_i \cdot u + \frac{9}{2} (c_i \cdot u)^2 - \frac{3}{2} u \cdot u  \big)
\end{equation}

The first order moments at a given cell are:
\begin{equation}
\begin{split}
\rho &= \sum_i{f_i} \\
\rho \, u_x &= \sum_i{f_i} c_i^x \\
\rho \, u_y &= \sum_i{f_i} c_i^y \\
\end{split}
\end{equation}

\begin{equation}
\begin{split}
\rho &=  f_0 + f_1 + f_2 + f_3 + f_4 + f_5 + f_6 + f_7 + f_8 \\
\rho \, u_x &= f_1 - f_3 + f_5 - f_6 - f_7 + f_8 \\
\rho \, u_y &= f_2 - f_4 + f_5 + f_6 - f_7 - f_8  \\
\end{split}
\end{equation}

\section{Units} 
In this paragraph and this paragraph only the superscript $^*$ denotes lattice units.
\subsection{Length, time, density}
The lattice units of length time and density are chosen to be equal to unity, which means that the conversion factors from lattice to physical units are 
\begin{equation}
\begin{split}
C_l &= \Delta x\\
C_t &= \Delta t\\
C_{\rho} &= \rho\\
\end{split}
\end{equation}

or equivalently
\begin{equation}
\begin{split}
l_{phys.} &= C_l \, l^*\\
t_{phys.} &= C_t \, t^*\\
\rho_{phys.} &= C_{\rho} \, \rho^*
\end{split}
\end{equation}

\subsection{Velocity}
The conversion factor for velocities is deduced from the factors set for length and time:
\begin{equation}
\begin{split}
C_u = C_l / C_t \\
U = C_u \, U^{*}
\end{split}
\end{equation}

\subsection{Pressure}
The lattice pressure $p^*$ is related to the lattice density $\rho^*$ through (\cite{TK} eq. 7.15)
\begin{equation}
p^* = \rho^* c_s^2\\
\end{equation}
And the conversion factor is 
\begin{equation}
C_{p} = \frac{C_\rho \, C_l^2}{C_t^2} 
\end{equation}


\subsection{Viscosity and Reynolds number}
The viscosity has the dimension of a square legnth to a time, so that 
\begin{equation}
\begin{split}
C_{\nu} = C_l^2 / C_t  = C_l \, C_u \\
\nu = C_{\nu} \, \nu^{*}
\end{split}
\end{equation}

For a given Reynolds number $Re$ and a reference length $L$, the physical and latice velocities are
\begin{equation}
\begin{split}
U = \frac{Re \, \nu}{L} \\
U^* = U / C_u
\end{split}
\end{equation}
Implictely, the Reynolds number is the same in both systems of units.

\section{Relaxation time} 
The convention for the relaxation parameter $\tau$ is the one used in TK 7.2.1, i.e. $\tau$ has the dimension of a time.

$\tau$ is the physical parameter and $\tau^*$ is a dimensionless parameter such that
\begin{equation}
\tau = \tau^* C_t
\end{equation}

The kinematic lattice viscosity is related to the relaxation parameter according to (TK 7.2.1.1) 
\begin{equation}
\begin{split}
\nu^* = c_s^2 (\tau^* - 1/2) \\
\nu = c_s^2 (\tau^* - 1/2) \frac{\Delta x^2}{\Delta t}
\end{split}
\end{equation}
This means that $\tau^*$, $\Delta x$ and $\Delta t$ are not independant.

\section{Parameter selection}
The choices made in foil5 follow the recommendations omade in TK §7.2.3.

Since the goal is to analyse an airfoil, the reference length is its chord which is set to $L=1$ physical unit, e.g. meters.

The problem is set up with:
\begin{itemize}
    \item The lattice width and height and the grid step $\Delta x$ which together define the number of cells
    \item The Reynolds number $Re$
    \item The kinematic viscosity $\nu$
\end{itemize}
The remaining parameters i.e the time step $\Delta t$ and the relaxation factor $\tau^*$ are free to choose to ensure stability and accuracy.

The first option is to set the relaxation factor: $\tau^*= 0.6$. This defines the time step (TK eq. 7.27):
$$\Delta t = c_s^2 (\tau^* - 1/2) \frac{\Delta x^2}{\nu}$$
and the lattice viscosity:
$$\nu^* = c_s^2 (\tau^* - 1/2)  = C_\nu \, \nu$$

The physical velocity $U$ is
$$U = \frac {Re \, \nu}{L}$$

and the lattice velocity is 
$$U^* = \frac{U} {C_u} = U \frac{\Delta t}{\Delta x}$$

Finally the lattice Reynolds number is defined by taking the lattice grid step as length scale (TK eq. 7.19):
$$Re_g = \frac{U_{max}^* \Delta x^*}{\nu^*} = \frac{U_{max}^*}{ c_s^2 (\tau^* - 1/2)} $$

\textbf{Application:}

$$Re = 1000$$
$$\nu = 1.5 \, 10^{-5}$$
$$\Delta x = 1/100 \, L$$ 
$$L = 1 (m) $$

With $c_s = 1/\sqrt{3}$ the rest of the parameters is deduced:
$$\Delta t = 0.222$$
$$U^* = 0.333$$
$$\nu^* = 0.0333$$
$$Re_g = 0.1$$


\subsection{BGK operator}
The relaxation factor for the BGK operator is $1/\tau$:
$$\Omega_i(f) = - \frac{f_i-f_i^{eq}}{\tau} \Delta t = - \frac{f_i-f_i^{eq}}{\tau^*}$$
\subsection{TRT and MRT}
The relaxation rate $\omega$ is defined as
$$\omega = \frac{1}{\tau}$$
The TRT operator relaxes even-order moments i.e. $\rho$ etc. with $\omega^+$ and odd-order moments i.e. $u$ with $\omega^-$.

$\omega^+$ is related to the kinematic shear viscosity (TK eq .10.47):
$$\nu = c_s^2 \big(\frac{1}{\omega^+ \Delta t} -\frac{1}{2} \big) C_\nu$$
$C_\nu$ is the conversion factor from lattice units to physical units in the case of viscosity:
$$C_\nu = \frac{\Delta x^2}{\Delta t}$$
$\omega^+$ is calculated using the previous equations:
$$\omega^+ \Delta t= \frac{1}{ (\nu/c_s^2/C_\nu + 1/2)}$$
$$\omega^+ \Delta t= \frac{2}{ (6\nu/C_\nu + 1)}$$
$$\omega^+ \Delta t= \frac{2 C_\nu}{ (6\nu + C_\nu )}$$

$\omega^-$ is a free parameter, which can be set indirectly using the "magic" parameter $\Lambda$ (TK eq. 10.43):
$$\Lambda = \big(\frac{1}{\omega^+ \Delta t} -\frac{1}{2} \big) \big(\frac{1}{\omega^- \Delta t} -\frac{1}{2} \big)$$
$\Lambda=1/4$ is the recommended value.
\subsection{Cascading}
The cascading or central moment collision operator is described in \cite{SL} and \cite{NP}. 
The two papers disagree on the last component of the moment variation vector $-36 u_x  u_y \dots$ in \cite{SL} and $+36 u_x  u_y \dots$ in \cite{NP}. \textbf{[investigate]}.


The effective relaxation parameter is
$$\omega_{eff} = \frac{\Delta t}{\Delta t_R} = \omega^+ \Delta t$$

\newpage
\begin{thebibliography}{9}

\bibitem{TK} 
Timm Krüger et al. 
\textit{The Lattice Boltzmann Method}, Springer, 2017.

\bibitem{ZH} 
Qisu Zou  and Xiaoyi He.
\textit{On pressure and velocity flow boundary conditions and bounceback for the lattice Boltzmann BGK model}, arXiv:comp-gas/9611001v1, Nov. 1996.
 
\bibitem{SL} 
S. Leclaire et al. 
\textit{Multiphase flow modeling of spinodal decomposition based on the cascaded lattice Boltzmann method}, Physica A 406 (2014) 307–319.

\bibitem{NP} 
N. Pellerin, S. Leclaire and M. Reggio. 
\textit{An implementation of the Spalart–Allmaras turbulence model in a multi-domain lattice Boltzmann method for solving turbulent airfoil flows}, Computers and Mathematics with Applications 70 (2015) 3001–3018.
 
\end{thebibliography}

\end{document}





