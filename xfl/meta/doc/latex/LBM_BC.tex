\documentclass[12pt]{article}
\usepackage[a4paper, portrait, margin=1in]{geometry}
\usepackage{mathtools}
\usepackage[export]{adjustbox}
\usepackage{color}
\usepackage{amsmath}
\usepackage{hyperref}
\usepackage{xfrac}

\graphicspath{{./figures/}}
\renewcommand{\familydefault}{\sfdefault}

 
\begin{document}

\title{Lattice Boltzmann Method - Implementation notes}
\author{techwinder}

\maketitle


\setlength{\parindent}{0pt}
\setlength{\parskip}{1em}

\section{Introduction}
This document is a summary of the notes taken during the implementation in foil5 of an LBM 2d flow solver for airfoils. 
\section{Lattice description}
foil5 implements the d2q9 operator, with the convention illustrated in figure \ref{naca12top}.

\begin{figure} 
\centering
\includegraphics[width=10cm]{d2q9.png}
\caption{Directions of distributions}
\label{naca12top}
\end{figure}

To simplify the imposition of boundary conditions, the velocity vectors have unit lengths 1 for the directions 1, 2, 3 and 4, and length $\sqrt{2}$ along the diagonal directions:
\begin{equation}
\label{Directions}
\begin{split}
c_0 &= (0,0) \\
c_1 &= (1,0) \\
c_2 &= (0,1) \\
c_3 &= (-1,0) \\
c_4 &= (0,-1) \\
c_5 &= (1,1) \\
c_6 &= (-1,1) \\
c_7 &= (-1,-1) \\
c_8 &= (1,-1) \\
\end{split}
\end{equation}

The weight factors of the distributions are:
\begin{equation}
\begin{split}
\label{Weight}
w_0 &= \frac{4}{9}\\
w_1 = w_2 = w_3 = w_4 &= \frac{1}{9}\\
w_5 = w_6 = w_7 = w_8 &= \frac{1}{36}
\end{split}
\end{equation}

The first order moments at a given cell are:
\begin{equation}
\begin{split}
\rho &= \sum_i{f_i} \\
\rho \, u_x &= \sum_i{f_i} c_i^x \\
\rho \, u_y &= \sum_i{f_i} c_i^y \\
\end{split}
\end{equation}
Developing the relations and introducing the values of the unit velocity vectors gives:
\begin{equation}
\begin{split}
\rho &= f_0 + f_1 + f_2 + f_3 + f_4 + f_5 + f_6 + f_7 + f_8 \\
\rho \, u_x &= f_1 - f_3  + f_5 - f_6 - f_7 + f_8 \\
\rho \, u_y &= f_2 - f_4 + f_5 + f_6 - f_7 - f_8
\end{split}
\end{equation}

Using the definitions of the equilibrium distribution:
$$f_i^{eq} = w_i \rho \, \big(1 + \frac{c_i \cdot u}{c_s^2} + \frac{(u \cdot c_i)^2}{2c_s^4} - \frac{u \cdot u}{2 c_s^2} \big)$$
$$f_i^{eq} =  w_i \, \rho \, \big( 1 + 3 \, c_i \cdot u + \frac{9}{2} (c_i \cdot u)^2 - \frac{3}{2} u \cdot u  \big)$$

\begin{equation}
\begin{split}
f_0^{eq} &= \frac{4}{9} \rho  \big(1 - \frac{3}{2} (u_x^2 + u_y^2)   \big)         \\
f_1^{eq} &= \frac{1}{9} \rho  \big(1 + 3 u_x + \frac{9}{2} u_x^2 - \frac{3}{2}  (u_x^2 + u_y^2) \big)            \\
f_2^{eq} &= \frac{1}{9} \rho  \big(1 + 3 u_y + \frac{9}{2} u_y^2 - \frac{3}{2}  (u_x^2 + u_y^2) \big)            \\ 
f_3^{eq} &= \frac{1}{9} \rho  \big(1 - 3 u_x + \frac{9}{2} u_x^2 - \frac{3}{2}  (u_x^2 + u_y^2) \big)          \\
f_4^{eq} &= \frac{1}{9} \rho  \big(1 - 3 u_y + \frac{9}{2} u_y^2 - \frac{3}{2}  (u_x^2 + u_y^2) \big)          \\
f_5^{eq} &= \frac{1}{36} \rho \big( 1 + 3 \, (u_x+u_y) + \frac{9}{2} (u_x+u_y)^2 - \frac{3}{2} (u_x^2 + u_y^2)  \big)            \\
f_6^{eq} &= \frac{1}{36} \rho \big( 1 + 3 \, (-u_x+u_y) + \frac{9}{2} (-u_x+u_y)^2 - \frac{3}{2} (u_x^2 + u_y^2)  \big)            \\
f_7^{eq} &= \frac{1}{36} \rho \big( 1 + 3 \, (-u_x-u_y) + \frac{9}{2} (-u_x-u_y)^2 - \frac{3}{2} (u_x^2 + u_y^2)  \big)            \\
f_8^{eq} &= \frac{1}{36} \rho \big( 1 + 3 \, (u_x-u_y) + \frac{9}{2} (u_x-u_y)^2 - \frac{3}{2} (u_x^2 + u_y^2) \big)            \\
\end{split}
\end{equation}
After simplification:
\begin{equation}
\begin{split}
f_0^{eq} &= \frac{4}{9} \rho  \big(1 - \frac{3}{2} (u_x^2 + u_y^2)   \big)         \\
f_1^{eq} &= \frac{1}{9} \rho  \big(1 + 3 u_x + 3 u_x^2 - \frac{3}{2} u_y^2 \big)            \\
f_2^{eq} &= \frac{1}{9} \rho  \big(1 + 3 u_y + 3 u_y^2 - \frac{3}{2} u_x^2 \big)            \\ 
f_3^{eq} &= \frac{1}{9} \rho  \big(1 - 3 u_x + 3 u_x^2 - \frac{3}{2} u_y^2 \big)          \\
f_4^{eq} &= \frac{1}{9} \rho  \big(1 - 3 u_y + 3 u_y^2 - \frac{3}{2} u_x^2 \big)          \\
f_5^{eq} &= \frac{1}{36} \rho \big( 1 + 3 \, (u_x+u_y) + 9 u_x u_y +3 (u_x^2 + u_y^2)  \big)            \\
f_6^{eq} &= \frac{1}{36} \rho \big( 1 - 3 \, (u_x-u_y) - 9 u_x u_y +3 (u_x^2 + u_y^2)  \big)            \\
f_7^{eq} &= \frac{1}{36} \rho \big( 1 - 3 \, (u_x+u_y) + 9 u_x u_y +3 (u_x^2 + u_y^2)  \big)            \\
f_8^{eq} &= \frac{1}{36} \rho \big( 1 + 3 \, (u_x-u_y) - 9 u_x u_y +3 (u_x^2 + u_y^2)  \big)            \\
\end{split}
\end{equation}

\section{Boundary conditions} 
This section uses the non-equilibrium bounce back (NEBB) method detailed in \cite{ZH}. 
The assumption made in \cite{ZH} is that the bounce-back principle applies at boundaries 
to the normal directions of the non-equilibrium part of the distributions. 

All combinations of conditions at the vertical and horizontal walls may not be compatible.
For instance a specification of a velocity at the inlet is incompatible with a no-slip condition on the horizontal wall 
since $u_x$ would be overspecified to incompatible values $0$ and $u_{x0}$.  Numerical experiments show that this 
leads to numerical instabilities.

Therefore only the following combinations are considered.
\begin{center}
\begin{tabular}{| c | c | c | c |  c |} 
 \hline
   & \textbf{Inlet} & \textbf{Horizontal} & \textbf{Outlet} & \textbf{Comment}\\  
 \hline
 1 &  velocity & velocity &  NRBC & pressure is unspecified \\ 
 \hline
 2 & velocity &  velocity & pressure & pressure waves \\ 
 \hline
 3 & pressure & velocity & pressure  & to be tested \\
 \hline
 4 & pressure & velocity & NRBC & to be tested\\
 \hline
 5 & NRBC & velocity & NRBC & to be tested\\
 \hline
 6 & pressure & slip or no-slip & pressure & velocity is unspecified \\
 \hline
\end{tabular}
\end{center}

 The only cases of interest for the calculation of a foil are the first four.

\section{Velocity B.C.}

\subsection{Inlet}
At the left inlet cells, $u_x$ and $u_y$ are specified and known.

After the collision and streaming steps, the equilibrium distributions are known and so are the distributions $f_2$, $f_3$, $f_4$, $f_6$, $f_7$.
The distributions $f_1$, $f_5$ and $f_8$ as well as the density $\rho$ are unknown. 

The NEBB condition at the vertical walls is
$$f_1-f_1^{eq} = f_3-f_3^{eq} $$
so that
$$f_1 = f_3 + \frac{2}{3} \rho u_x$$

$\rho$, is determined by eliminating the unknowns variables from the expressions of the moments:
$$\rho = \frac{1}{(1-u_x) } \big(f_0 + f_2  + f_4 + 2 (f_3 + f_6 + f_7) \big)$$

$f_5$ and $f_8$ are determined by eliminating $f_1$ from the expressions of moments:
\begin{equation}
\begin{split}
f_5 + f_8 &= \rho \, u_x - f_1 + f_3 + f_6 + f_7 = \rho \, u_x -\frac{2}{3} \rho u_x + f_6 + f_7  = \frac{1}{3} \rho u_x + f_6 + f_7 \\
f_5 - f_8 &= \rho \, u_y - f_2 + f_4 - f_6 + f_7 = \rho \, u_y - f_6 + f_7 
\end{split}
\end{equation}

\begin{equation}
\begin{split}
f_5 &= \frac{1}{2}\big( \frac{1}{3} \rho u_x + f_6 + f_7  + \rho u_y - f_2 + f_4 - f_6 + f_7  \big) \\
f_5 &= \frac{1}{2}\big( \frac{1}{3} \rho u_x + \rho u_y - f_2 + f_4 + 2 f_7  \big) \\
\end{split}
\end{equation}
and
\begin{equation}
\begin{split}
f_8 &= \frac{1}{2} \big( \frac{1}{3} \rho u_x + f_6 + f_7 -\rho u_y + f_2 - f_4 + f_6 - f_7 \big) \\
\end{split}
\end{equation}
Finally:
\begin{equation}
\begin{split}
\rho &= \frac{1}{(1-u_x) } \big(f_0 + f_2  + f_4 + 2 (f_3 + f_6 + f_7) \big) \\
f_1 &= f_3 + \frac{2}{3} \rho u_x \\
f_5 &= f_7 -\frac{1}{2} (f_2 - f_4) + \frac{1}{6} \rho u_x + \frac{1}{2} \rho u_y   \\
f_8 &= f_6 +\frac{1}{2} (f_2 - f_4) + \frac{1}{6} \rho u_x - \frac{1}{2} \rho u_y  
\end{split}
\end{equation}
This is identical to TK eq. 5.49 to 5.52 except for a typo in their last equation.


\subsection{Outlet}
At the right outlet cells, $u_x$ and $u_y$ are specified and known.

After the collision and streaming steps, the equilibrium distributions are known and so are the distributions $f_1$, $f_2$, $f_4$, $f_5$, $f_8$.
The distributions $f_6$, $f_3$ and $f_7	$ as well as the density $\rho$ are unknown. 

The NEBB condition at the vertical walls is
$$f_3-f_3^{eq} = f_1-f_1^{eq}$$
so that
$$f_3 = f_1 - \frac{2}{3} \rho u_x$$

$\rho$, is determined by eliminating the unknowns variables from the expressions of the moments:
\begin{equation}
\begin{split}
\rho &= f_0 + f_1 + f_2 + f_3 + f_4 + f_5 + f_6 + f_7 + f_8 \\
\rho \, u_x &= f_1 - f_3  + f_5 - f_6 - f_7 + f_8 \\
\rho \, u_y &= f_2 - f_4 + f_5 + f_6 - f_7 - f_8
\end{split}
\end{equation}

$$\rho = \frac{1}{1+u_x} \big(f_0 + f_2  + f_4 + 2 (f_1 + f_5 + f_8) \big)$$

$f_6$ and $f_7$ are determined by eliminating $f_1$ from the expressions of moments:
\begin{equation}
\begin{split}
f_6 + f_7 &= -\rho \, u_x + f_1 - f_3 + f_5 + f_8  \\
f_6 - f_7 &= \rho \, u_y - f_2 + f_4 - f_5 + f_8  
\end{split}
\end{equation}

\begin{equation}
\begin{split}
f_6 + f_7 &= -\rho u_x +\frac{2}{3} \rho u_x + f_5 + f_8  = -\frac{1}{3} \rho u_x + f_5 + f_8 \\
f_6 - f_7 &= \rho \, u_y - (f_2 - f_4) - f_5 + f_8  
\end{split}
\end{equation}

\begin{equation}
\begin{split}
f_6 &= \frac{1}{2} \big(-\frac{1}{3} \rho u_x + f_5 + f_8 + \rho u_y - (f_2-f_4) - f_5 + f_8 \big)\\ 
f_7 &= \frac{1}{2} \big(-\frac{1}{3} \rho u_x + f_5 + f_8 - \rho u_y + (f_2-f_4) + f_5 - f_8 \big)  
\end{split}
\end{equation}

Finally:
\begin{equation}
\begin{split}
\rho &= \frac{1}{1+u_x} \big(f_0 + f_2  + f_4 + 2 (f_1 + f_5 + f_8) \big) \\
f_3 &= f_1 - \frac{2}{3} \rho u_x \\
f_6 &= f_8 -\frac{1}{2} (f_2 - f_4) - \frac{1}{6} \rho u_x + \frac{1}{2} \rho u_y   \\
f_7 &= f_5 +\frac{1}{2} (f_2 - f_4) - \frac{1}{6} \rho u_x - \frac{1}{2} \rho u_y  
\end{split}
\end{equation}



\subsection{Bottom}
On the bottom boundary, $u_x$ and $u_y$ are specified and known.

After the collision and streaming steps, the distributions $f_1$, $f_3$, $f_4$, $f_7$, $f_8$ are known.
The distributions $f_6$, $f_2$ and $f_5$ as well as the density $\rho$ are unknown. 

The NEBB condition is
$$f_2-f_2^{eq} = f_4-f_4^{eq}$$

so that
$$f_2 = f_4 + (f_2^{eq}-f_4^{eq}) = f_4 + \frac{2}{3} \rho u_y$$

\begin{equation}
\begin{split}
\rho &= f_0 + f_1 + f_2 + f_3 + f_4 + f_5 + f_6 + f_7 + f_8 \\
\rho \, u_x &= f_1 - f_3  + f_5 - f_6 - f_7 + f_8 \\
\rho \, u_y &= f_2 - f_4 + f_5 + f_6 - f_7 - f_8
\end{split}
\end{equation}

$\rho$, is determined by eliminating the unknown variables from the expressions of the moments:
$$\rho = \frac{1}{(1-u_y) } \big(f_0 + f_1  + f_3 + 2 (f_4 + f_7 + f_8) \big)$$

$f_5$ and $f_6$ are determined from the expressions of moments:
\begin{equation}
\begin{split}
f_5  &= \frac{1}{2} \big( \rho (u_x+u_y) - f_1 - f_2 + f_3 + f_4 + 2 f_7 \big)\\
f_6  &= \frac{1}{2} \big( \rho (u_y-u_x) + f_1 - f_2 - f_3 + f_4 + 2 f_8 \big) 
\end{split}
\end{equation}

\begin{equation}
\begin{split}
f_5  &= \frac{1}{2} \big( \rho (u_x+u_y) -\frac{2}{3} \rho u_y - f_1 + f_3  + 2 f_7 \big)\\
f_6  &= \frac{1}{2} \big( \rho (u_y-u_x) -\frac{2}{3} \rho u_y + f_1 - f_3  + 2 f_8 \big) 
\end{split}
\end{equation}

Finally:
\begin{equation}
\begin{split}
\rho &= \frac{1}{(1-u_y) } \big(f_0 + f_1  + f_3 + 2 (f_4 + f_7 + f_8) \big) \\
f_2 &= f_4 + \frac{2}{3} \rho u_y \\
f_5 &= f_7 -\frac{1}{2} (f_1 - f_3) + \frac{1}{2} \rho u_x + \frac{1}{6} \rho u_y \\
f_6 &= f_8 +\frac{1}{2} (f_1 - f_3) - \frac{1}{2} \rho u_x + \frac{1}{6} \rho u_y 
\end{split}
\end{equation}


\subsection{Top}
On the top boundary, $u_x$ and $u_y$ are specified and known.

After the collision and streaming steps, the distributions $f_1$, $f_3$, $f_6$, $f_2$, $f_5$ are known.
The distributions $f_7$, $f_4$ and $f_8$ as well as the density $\rho$ are unknown. 

The NEBB conditions is
$$f_4-f_4^{eq} = f_2-f_2^{eq}$$

so that
$$f_4 = f_2 - \frac{2}{3} \rho u_y$$

$\rho$, is determined by eliminating the unknowns variables from the expressions of the moments:
$$\rho = \frac{1}{(1+u_y) } \big(f_0 + f_1  + f_3 + 2 (f_2 + f_5 + f_6) \big)$$

$f_7$ and $f_8$ are determined from the expressions of moments:
\begin{equation}
\label{Moment_eq}
\begin{split}
\rho &= f_0 + f_1 + f_2 + f_3 + f_4 + f_5 + f_6 + f_7 + f_8 \\
\rho \, u_x &= f_1 - f_3  + f_5 - f_6 - f_7 + f_8 \\
\rho \, u_y &= f_2 - f_4 + f_5 + f_6 - f_7 - f_8
\end{split}
\end{equation}
\begin{equation}
\begin{split}
f_7  &= \frac{1}{2} \big(-\rho (u_x+u_y) + f_1 + f_2 - f_3 - f_4 + 2 f_5 \big)\\
f_8  &= \frac{1}{2} \big( \rho (u_x-u_y) - f_1 + f_2 + f_3 - f_4 + 2 f_6 \big) 
\end{split}
\end{equation}

\begin{equation}
\begin{split}
f_7  &= f_5 - \frac{1}{2}  \rho (u_x+u_y) +\frac{1}{3} \rho u_y + \frac{1}{2} (f_1 - f_3) \\
f_8  &= f_6 + \frac{1}{2}  \rho (u_x-u_y) +\frac{1}{3} \rho u_y - \frac{1}{2} (f_1 - f_3)   
\end{split}
\end{equation}

Finally (TK 6.58):
\begin{equation}
\begin{split}
\rho &= \frac{1}{(1+u_y) } \big(f_0 + f_1  + f_3 + 2 (f_2 + f_5 + f_6) \big)\\
f_4 &= f_2 - \frac{2}{3} \rho u_y \\
f_7 &= f_5 +\frac{1}{2} (f_1 - f_3) - \frac{1}{6} \rho u_y - \frac{1}{2} \rho u_x   \\
f_8 &= f_6 -\frac{1}{2} (f_1 - f_3) - \frac{1}{6} \rho u_y + \frac{1}{2} \rho u_x 
\end{split}
\end{equation}

\subsection{Corners}

\subsubsection{Inlet bottom}

After the collision and streaming steps, the equilibrium distributions are known and so are the distributions $f_3$, $f_4$, and $f_7$.

The velocities $u_x$ and  $u_y$ are specified and known. 

The distributions $f_0$, $f_1$, $f_2$, $f_5$, $f_6$ and $f_8$ as well as the density $\rho$ are the seven unknown variables.

The bounce-back rule for the non-equilibrium part of the distributions normal to the inlet and to the boundary are:
\begin{equation}
\begin{split}
f_1-f_1^{eq} &= f_3-f_3^{eq} \\
f_2-f_2^{eq} &= f_4-f_4^{eq} \\
f_5-f_5^{eq} &= f_7-f_7^{eq}
\end{split}
\end{equation}

$f_1$, $f_2$ and $f_5$ are determined using the definitions of the equilibrium distribution \eqref{Eq_distrib}
\begin{equation}
\begin{split}
f_1 &= f_3 \frac{2}{3} \rho u_x    \\
f_2 &= f_4 \frac{2}{3} \rho u_y    \\
f_5 &= f_7 \frac{1}{6} \rho (u_x+u_y)    \\
\end{split}
\end{equation}

$f_6$, $f_8$ and $\rho$ are determined from the momentum expressions
\begin{equation}
\begin{split}
\rho &=  f_0 + f_1 + f_2 + f_3 + f_4 + f_5 + f_6 + f_7 + f_8 \\
\rho \, u_x &= f_1 - f_3 + f_5 - f_6 - f_7 + f_8 \\
\rho \, u_y &= f_2 - f_4 + f_5 + f_6 - f_7 - f_8  \\
\end{split}
\end{equation}
which simplify to
\begin{equation}
\begin{split}
\rho &= f_0 + f_1 + f_2 + f_3 + f_4 + f_5 + f_6 + f_7 + f_8    \\
\frac{1}{3}\rho \, u_x &= f_5 - f_6 - f_7 + f_8  = - f_6 + f_8 +\frac{1}{6} \rho (u_x+u_y)                           \\
\frac{1}{3}\rho \, u_y &= f_5 + f_6 - f_7 - f_8  = + f_6 - f_8 +\frac{1}{6} \rho (u_x+u_y)  
\end{split}
\end{equation}
The last two equations are identical.
\begin{equation}
\begin{split}
\rho &= f_0 + f_1 + f_2 + f_3 + f_4 + f_5 + f_6 + f_7 + f_8    \\
\frac{1}{6}\rho (u_x-u_y) &= - f_6 + f_8                            \\
\end{split}
\end{equation}
The system misses one equation to be invertible (cf. TK Eq. 5.57).

The method applied in \cite{ZH} is to assume uniform density locally, so that $\rho$ at the inlet bottom node can be taken equal to the density of its neighboring flow node.

With $\rho$ known, $f_6$ and $f_8$ can be determined as follows:
\begin{equation}
\begin{split}
f_6 &= \frac{1}{2} \big( \rho-(f_0+f_1+f_2+f_3+f_4+f_5+f_7) -\frac{1}{6} (u_x-u_y) \big)\\
f_8 &= \frac{1}{2} \big( \rho-(f_0+f_1+f_2+f_3+f_4+f_5+f_7) +\frac{1}{6} (u_x-u_y) \big)\\
f_8 &= f_6 + \frac{1}{6} \rho (u_x-u_y) 
\end{split}
\end{equation}

and finally $f_0$ is set to ensure mass conservation:
\begin{equation}
f_0 = \rho - (f_1 + f_2 + f_3 + f_4 + f_5 + f_6 + f_7 + f_8  ) \\
\end{equation}


\subsubsection{Inlet top}

After the collision and streaming steps, the equilibrium distributions are known and so are the distributions $f_2$, $f_3$, and $f_6$.

The velocity $u_x$ and $u_y$ are specified. $u_x$ is either $0$ if or $u_{x0}$. 

The distributions $f_0$,  $f_1$, $f_4$, $f_5$, $f_7$ and $f_8$ as well as the density $\rho$ are the unknown variables.

The bounce-back rule for the non-equilibrium part of the distributions normal to the inlet and to the boundary are:
\begin{equation}
\begin{split}
f_1-f_1^{eq} &= f_3-f_3^{eq} \\
f_2-f_2^{eq} &= f_4-f_4^{eq} \\
f_6-f_6^{eq} &= f_8-f_8^{eq}
\end{split}
\end{equation}

Using the definitions of the equilibrium distributions:
\begin{equation}
\begin{split}
f_1 &= f_3 + \frac{2}{3} \rho u_x            \\ 
f_4 &= f_2 -\frac{2}{3} \rho u_y            \\
f_8 &= f_6 + \frac{1}{6} \rho (u_x-u_y)        \\
\end{split}
\end{equation}


$\rho$, $f_5$ and $f_7$ are determined from the momentum expressions 
\begin{equation}
\begin{split}
\rho &=  f_0 + f_1 + f_2 + f_3 + f_4 + f_5 + f_6 + f_7 + f_8 \\
\rho u_x &= f_1 - f_3 + f_5 - f_6 - f_7 + f_8 \\
\rho u_y &= f_2 - f_4 + f_5 + f_6 - f_7 - f_8  \\
\end{split}
\end{equation}
which simplify to:
\begin{equation}
\begin{split}
\rho &= f_0 + f_1 + f_2 + f_3 + f_4 + f_5 + f_6 + f_7 + f_8    \\
\frac{1}{3}\rho  u_x &= f_5 - f_6 - f_7 + f_8                            \\
\frac{1}{3}\rho  u_y &= f_5 + f_6 - f_7 - f_8
\end{split}
\end{equation}

The last two equations are identical.
\begin{equation}
\begin{split}
\rho &= f_0 + f_1 + f_2 + f_3 + f_4 + f_5 + f_6 + f_7 + f_8    \\
\frac{1}{6}\rho  (u_x+u_y) &= f_5 - f_7 
\end{split}
\end{equation}

As for the inlet bottom corner, uniform local density is assumed and the value of $\rho$ is set equal to that of its neighbour cell. With $\rho$ known, $f_5$ and $f_7$ can be determined as follows:
\begin{equation}
\begin{split}
f_5 + f_7 &= \rho-(f_0+f_1+f_2+f_3+f_4+f_6+f_8) \\
f_5 - f_7 &= \frac{1}{6} \rho (u_x +u_y)
\end{split}
\end{equation}

finally
\begin{equation}
\begin{split}
f_5 &= \frac{1}{2} ( \rho-(f_0+f_1+f_2+f_3+f_4+f_6+f_8) +\frac{1}{6} (u_x+u_y))\\
f_7 &= \frac{1}{2} ( \rho-(f_0+f_1+f_2+f_3+f_4+f_6+f_8) -	\frac{1}{6} (u_x+u_y))\\
f_5 - f_7 &= \frac{1}{6} \rho (u_x +u_y)
\end{split}
\end{equation}

\subsubsection{Outlet bottom}

After the collision and streaming steps, the equilibrium distributions are known and so are the distributions $f_1$, $f_4$, and $f_8$.

The velocities $u_x$ and $u_y$ are specified and known.

The distributions $f_0$, $f_2$, $f_3$, $f_5$, $f_6$ and $f_7$ as well as the density $\rho$ are the seven unknown variables.

The bounce-back rule for the non-equilibrium part of the distributions normal to the inlet and to the boundary are:
\begin{equation}
\begin{split}
f_3-f_3^{eq} &= f_1-f_1^{eq} \\
f_2-f_2^{eq} &= f_4-f_4^{eq} \\
f_6-f_6^{eq} &= f_8-f_8^{eq}
\end{split}
\end{equation}

$f_2$, $f_3$ and $f_6$ are determined using the definitions of the equilibrium distribution \eqref{Eq_distrib}
\begin{equation}
\begin{split}
f_2 &= f_4 + \frac{2}{3} \rho u_y    \\
f_3 &= f_1 - \frac{2}{3} \rho u_x    \\
f_6 &= f_8 - \frac{1}{6} \rho (u_x-u_y)    \\
\end{split}
\end{equation}

$f_5$, $f_7$ and $\rho$ are determined from the expressions of moments
\begin{equation}
\begin{split}
\rho &= f_0 + f_1 + f_2 + f_3 + f_4 + f_5 + f_6 + f_7 + f_8 \\
\rho \, u_x &= f_1 - f_3  + f_5 - f_6 - f_7 + f_8 \\
\rho \, u_y &= f_2 - f_4 + f_5 + f_6 - f_7 - f_8
\end{split}
\end{equation}

which simplify to
\begin{equation}
\begin{split}
\rho &= f_0 + f_1 + f_2 + f_3 + f_4 + f_5 + f_6 + f_7 + f_8    \\
\frac{1}{3}\rho u_x &= f_5 - f_6 - f_7 + f_8  = f_5 - f_7 +\frac{1}{6} \rho (u_x-u_y)  \\
\frac{1}{3}\rho u_y &= f_5 + f_6 - f_7 - f_8  = f_5 - f_7 -\frac{1}{6} \rho (u_x-u_y)  
\end{split}
\end{equation}
The last two equations are identical.
\begin{equation}
\begin{split}
\rho &= f_0 + f_1 + f_2 + f_3 + f_4 + f_5 + f_6 + f_7 + f_8    \\
\frac{1}{6}\rho (u_x+u_y) &= f_5 - f_7                            \\
\end{split}
\end{equation}
The system misses one equation to be invertible (cf. TK Eq. 5.57).

The method applied in \cite{ZH} is to assume uniform density locally, so that $\rho$ at the inlet bottom node can be taken equal to the density of its neighboring flow node.

With $\rho$ known, $f_5$ and $f_7$ can be determined as follows:
\begin{equation}
\begin{split}
f_5 &= \frac{1}{2} \big( \rho-(f_0+f_1+f_2+f_3+f_4+f_6+f_8) +\frac{1}{6} \rho(u_x+u_y) \big)\\
f_7 &= \frac{1}{2} \big( \rho-(f_0+f_1+f_2+f_3+f_4+f_6+f_8) -\frac{1}{6} \rho(u_x+u_y) \big)\\
f_7 &= f_5 - \frac{1}{6} \rho (u_x+u_y) 
\end{split}
\end{equation}

and finally $f_0$ is set to ensure mass conservation:
\begin{equation}
f_0 = \rho - (f_1 + f_2 + f_3 + f_4 + f_5 + f_6 + f_7 + f_8  ) \\
\end{equation}


\subsubsection{Outlet top}


After the collision and streaming steps, the equilibrium distributions are known and so are the distributions $f_1$, $f_2$, and $f_5$.

The velocities $u_x$ and $u_y$ are specified and known.

The distributions $f_0$, $f_3$, $f_4$, $f_6$, $f_7$ and $f_8$ as well as the density $\rho$ are the seven unknown variables.

The bounce-back rule for the non-equilibrium part of the distributions normal to the inlet and to the boundary are:
\begin{equation}
\begin{split}
f_1-f_1^{eq} &= f_3-f_3^{eq} \\
f_4-f_4^{eq} &= f_2-f_2^{eq} \\
f_7-f_7^{eq} &= f_5-f_5^{eq}
\end{split}
\end{equation}

$f_3$, $f_4$ and $f_7$ are determined using the definitions of the equilibrium distribution \begin{equation}
\begin{split}
f_3 &= f_1 - \frac{2}{3} \rho u_x    \\
f_4 &= f_2 - \frac{2}{3} \rho u_y    \\
f_7 &= f_5 - \frac{1}{6} \rho (u_x+u_y)    \\
\end{split}
\end{equation}

$f_6$, $f_8$ and $\rho$ are determined from the expressions of moments
\begin{equation}
\begin{split}
\rho &= f_0 + f_1 + f_2 + f_3 + f_4 + f_5 + f_6 + f_7 + f_8 \\
\rho \, u_x &= f_1 - f_3  + f_5 - f_6 - f_7 + f_8 \\
\rho \, u_y &= f_2 - f_4 + f_5 + f_6 - f_7 - f_8
\end{split}
\end{equation}

which simplify to
\begin{equation}
\begin{split}
\rho &= f_0 + f_1 + f_2 + f_3 + f_4 + f_5 + f_6 + f_7 + f_8    \\
\frac{1}{3}\rho u_x &= f_5 - f_6 - f_7 + f_8  = -f_6 + f_8 +\frac{1}{6} \rho (u_x+u_y)  \\
\frac{1}{3}\rho u_y &= f_5 + f_6 - f_7 - f_8  = f_6 - f_8 +\frac{1}{6} \rho (u_x+u_y)  
\end{split}
\end{equation}
The last two equations are identical.
\begin{equation}
\begin{split}
\rho &= f_0 + f_1 + f_2 + f_3 + f_4 + f_5 + f_6 + f_7 + f_8    \\
\frac{1}{6}\rho (-u_x+u_y) &= f_6 - f_8                            \\
\end{split}
\end{equation}
The system misses one equation to be invertible (cf. TK Eq. 5.57).

The method applied in \cite{ZH} is to assume uniform density locally, so that $\rho$ at the inlet bottom node can be taken equal to the density of its neighboring flow node.

With $\rho$ known, $f_6$ and $f_8$ can be determined as follows:
\begin{equation}
\begin{split}
f_6 &= \frac{1}{2} \big( \rho-(f_0+f_1+f_2+f_3+f_4+f_5+f_7) +\frac{1}{6} \rho (-u_x+u_y) \big)\\
f_8 &= \frac{1}{2} \big( \rho-(f_0+f_1+f_2+f_3+f_4+f_5+f_7) -\frac{1}{6} \rho (-u_x+u_y) \big)\\
f_8 &= f_6 - \frac{1}{6} \rho (-u_x+u_y) 
\end{split}
\end{equation}

and finally $f_0$ is set to ensure mass conservation:
\begin{equation}
f_0 = \rho - (f_1 + f_2 + f_3 + f_4 + f_5 + f_6 + f_7 + f_8  ) \\
\end{equation}


\section{Density B.C.}
\subsection{Inlet}
At the left inlet cells, $\rho_{in}$ is specified and $u_y=0$.


After the collision and streaming steps, the equilibrium distributions are known and so are the distributions $f_2$, $f_3$, $f_4$, $f_6$, $f_7$.
The distributions $f_1$, $f_5$ and $f_8$ as well as the velocity $u_x$ are unknown. 


Using the definitions of the equilibrium distribution and moment equations \eqref{Moment_eq}:
\begin{equation}
\begin{split}
f_1 + f_5 + f_8 = \rho_{in} - (f_0 + f_2 + f_3 +f_4 + f_6 + f_7 ) \\
f_1 + f_5 + f_8 = \rho_{in} u_x  + (f_3 + f_6 + f_7 ) \\
\rho_{in} \, u_y = 0 = f_2 - f_4 + f_5 + f_6 - f_7 - f_8
\end{split}
\end{equation}

$u_x$ is deduced from the first two equations
\begin{equation}
u_x = 1- \frac{f_0 + f_2 + f_4 + 2(f_3+f_6 + f_7)}{\rho_{in}}
\end{equation}

The NEBB condition determines $f_1$:
\begin{equation}
f_1 = f_1^{eq} + f_3-f_3^{eq} = f_3 + \frac{2}{3} \rho_{in} u_x
\end{equation}

$f_5$ and $f_8$ are determined by eliminating $f_1$ from the momentum equations:
\begin{equation}
\begin{split}
f_5 = f_7 - \frac{1}{2} (f_2-f_4)  + \frac{1}{6} \rho_{in} u_x\\
f_8 = f_6 + \frac{1}{2} (f_2-f_4)  + \frac{1}{6} \rho_{in} u_x\\
\end{split}
\end{equation}

\subsection{Outlet}
At the right outlet cells, $\rho_{out}$ is specified and $u_y=0$.

After the collision and streaming steps, the equilibrium distributions are known and so are the distributions $f_1$, $f_2$, $f_4$, $f_5$, $f_8$.

The distributions $f_3$, $f_6$ and $f_7$ as well as the velocity $u_x$ are unknown. 

Using the definitions of the equilibrium distribution and moment equations \eqref{Moment_eq}:
\begin{equation}
\begin{split}
f_3 + f_6 + f_7 = \rho_{out} - (f_0 + f_1 + f_2 + f_4 + f_5 + f_8 ) \\
f_3 + f_6 + f_7 = -\rho_{out} u_x  + (f_1 + f_5 + f_8) \\
\rho_{out} \, u_y = 0 = f_2 - f_4 + f_5 + f_6 - f_7 - f_8
\end{split}
\end{equation}

or
\begin{equation}
\begin{split}
f_3 + f_6 + f_7 = \rho_{out} - (f_0 + f_1 + f_2 + f_4 + f_5 + f_8 ) \\
f_3 + f_6 + f_7 = -\rho_{out} u_x  + (f_1 + f_5 + f_8) \\
f_6 - f_7  = -f_2 + f_4 - f_5 + f_8
\end{split}
\end{equation}

$u_x$ is deduced from the first two equations
\begin{equation}
0 = \rho_{out} (1+u_x) - (f_0 + f_2 + f_4 + 2(f_1+f_5 + f_8))
\end{equation}
\begin{equation}
u_x = -1 + \frac{f_0 + f_2 + f_4 + 2(f_1+f_5 + f_8)}{\rho_{out}}
\end{equation}

The NEBB condition determines $f_3$:
\begin{equation}
f_3 =  f_1- \frac{2}{3} \rho_{out} u_x
\end{equation}


\begin{equation}
\begin{split}
f_6 + f_7 = -\rho_{out} u_x  + (f_1 - f_3 + f_5 + f_8) \\
f_6 - f_7  = -f_2 + f_4 - f_5 + f_8
\end{split}
\end{equation}


\begin{equation}
\begin{split}
f_6 + f_7 = -\rho_{out} u_x  + \frac{2}{3} \rho_{out} u_x + (f_5 + f_8) \\
f_6 - f_7  = -f_2 + f_4 - f_5 + f_8
\end{split}
\end{equation}


\begin{equation}
\begin{split}
f_6 + f_7 = - \frac{1}{3} \rho_{out} u_x + (f_5 + f_8) \\
f_6 - f_7  = -f_2 + f_4 - f_5 + f_8
\end{split}
\end{equation}


\begin{equation}
\begin{split}
2 f_6 = - \frac{1}{3} \rho_{out} u_x + f_5 + f_8 -f_2 + f_4 - f_5 + f_8\\
2 f_6 = 2 f_8 - \frac{1}{3} \rho_{out} u_x  -f_2 + f_4 \\
f_6 = f_8 - \frac{1}{6} \rho_{out} u_x -\frac{1}{2}( f_2 - f_4 ) \\
\end{split}
\end{equation}


\begin{equation}
\begin{split}
2 f_7 = - \frac{1}{3} \rho_{out} u_x + f_5 + f_8 + f_2 - f_4 + f_5 - f_8\\
2 f_7 = 2 f_5 - \frac{1}{3} \rho_{out} u_x + f_5 + f_2 - f_4 + f_5 \\
f_7 =  f_5 - \frac{1}{6} \rho_{out} u_x + \frac{1}{2} (f_2 - f_4) \\
\end{split}
\end{equation}


Summing up, the relations are the same as for the inlet:
\begin{equation}
\begin{split}
f_6 = f_8 - \frac{1}{6} \rho_{out} u_x -\frac{1}{2}( f_2 - f_4 ) \\
f_7 =  f_5 - \frac{1}{6} \rho_{out} u_x + \frac{1}{2} (f_2 - f_4) \\
\end{split}
\end{equation}

\subsection{Corner points}
Using \cite{ZH} equations 24 and 25.

\subsubsection{Inlet bottom}

$\rho_{in}$ is specified and known.
$f_3$, $f_4$ and $f_7$ are known after streaming from adjacent nodes.
In the no-slip case, $u_x=u_y=0$.

The unknowns are $f_1$, $f_2$ $f_5$, $f_6$ and $f_8$.

Using NEBB at the inlet and taking into account that $u_x$ and $u_y$ are $0$:
\begin{equation}
f_1 = f_3+(f_1^{eq}-f_3^{eq}) = f_3
\end{equation}
Similarly using NEBB condition on the bottom wall:
\begin{equation}
f_2 = f_4+(f_4^{eq}-f_2^{eq}) = f_4
\end{equation}

Using \eqref{Inlet_rho}
\begin{equation}
\begin{split}
f_1 + f_5 + f_8 = \rho_{in} - (f_0 + f_2 + f_3 +f_4 + f_6 + f_7 ) \\
f_1 + f_5 + f_8 = \rho_{in} u_x  + (f_3 + f_6 + f_7 ) \\
\rho_{in} \, u_y = 0 = f_2 - f_4 + f_5 + f_6 - f_7 - f_8
\end{split}
\end{equation}

\begin{equation}
\begin{split}
f_1 + f_5 + f_8 = \rho_{in} - (f_0 + f_2 + f_3 +f_4 + f_6 + f_7 ) \\
f_1 + f_5 + f_8 =  f_3 + f_6 + f_7  \\
f_2 - f_4 + f_5 + f_6 - f_7 - f_8 = 0
\end{split}
\end{equation}

\begin{equation}
\begin{split}
f_5 + f_8 = \rho_{in} - (f_0 + f_1 + f_2 + f_3 +f_4 + f_6 + f_7 ) \\
f_5 + f_8 - f_6 - f_7 = 0 \\
f_5 + f_6 - f_7 - f_8 = 0
\end{split}
\end{equation}

\begin{equation}
\begin{split}
f_5 = f_7 \\
f_6 + f_8 = \rho_{in} - (f_0 + f_1 + f_2 + f_3 +f_4 + f_5 + f_7 ) \\
f_6 = f_8 
\end{split}
\end{equation}

\begin{equation}
\begin{split}
f_5 = f_7 \\
f_6 = f_8 = \frac{1}{2} (\rho_{in} - (f_0 + f_1 + f_2 + f_3 +f_4 + f_5 + f_7 ) \\
\end{split}
\end{equation}


\subsubsection{Inlet top}

$\rho_{in}$ is specified and known.
$f_2$, $f_3$ and $f_6$ are known after streaming from adjacent nodes.
In the no-slip case, $u_x=u_y=0$.

The unknowns are $f_1$, $f_4$ $f_5$, $f_7$ and $f_8$.

Using NEBB at the inlet and taking into account that $u_x$ and $u_y$ are $0$:
\begin{equation}
f_1 = f_3+(f_1^{eq}-f_3^{eq}) = f_3
\end{equation}
Similarly using NEBB condition on the bottom wall:
\begin{equation}
f_4 = f_2+(f_2^{eq}-f_4^{eq}) = f_2
\end{equation}

The same development as before leads to

\begin{equation}
\begin{split}
f_1 + f_5 + f_8 &= \rho_{in} - (f_0 + f_2 + f_3 +f_4 + f_6 + f_7 ) \\
f_5 &= f_7 \\
f_8 &= f_6
\end{split}
\end{equation}

\begin{equation}
\begin{split}
f_5 + f_7 = 2 f_5 = 2 f_7 &=  \rho_{in} - (f_0 + f_1 + f_2 + f_3 +f_4 + f_6 + f_8) \\
f_8 &= f_6
\end{split}
\end{equation}

\begin{equation}
\begin{split}
f_5 &= f_7  = \frac{1}{2} (\rho_{in} - (f_0 + f_1 + f_2 + f_3 +f_4 + f_6 + f_8)) \\
f_8 &= f_6
\end{split}
\end{equation}

\subsubsection{Outlet bottom}

$\rho_{out}$ is specified and known.
$f_1$, $f_4$ and $f_8$ are known after streaming from adjacent nodes.
In the no-slip case, $u_x=u_y=0$.

The unknowns are $f_2$, $f_3$ $f_5$, $f_6$ and $f_7$.

Using NEBB at the inlet and taking into account that $u_x$ and $u_y$ are $0$:
\begin{equation}
\begin{split}
f_3 &= f_1+(f_3^{eq}-f_1^{eq}) = f_1 \\
f_2 &= f_4+(f_4^{eq}-f_2^{eq}) = f_4
\end{split}
\end{equation}

Using \eqref{Inlet_rho}
\begin{equation}
\begin{split}
f_1 + f_5 + f_8 &= \rho_{out} - (f_0 + f_2 + f_3 +f_4 + f_6 + f_7 ) \\
f_1 + f_5 + f_8 &= f_3 + f_6 + f_7  \\
0 &= f_2 - f_4 + f_5 + f_6 - f_7 - f_8
\end{split}
\end{equation}

\begin{equation}
\begin{split}
f_5 + f_7& = \rho_{out} - (f_0 + f_1 + f_2 + f_3 +f_4 + f_6 + f_8 ) \\
f_5 + f_8 - f_6 - f_7 &= 0\\
f_5 + f_6 - f_7 - f_8 &= 0 \\
\end{split}
\end{equation}

\begin{equation}
\begin{split}
f_5 + f_7 &= \rho_{out} - (f_0 + f_1 + f_2 + f_3 +f_4 + f_6 + f_8 ) \\
f_5 &= f_7 \\
f_6 &= f_8 \\
\end{split}
\end{equation}

\begin{equation}
\begin{split}
f_5 + f_7 &= 2f_5 = 2f_7 = \rho_{out} - (f_0 + f_1 + f_2 + f_3 +f_4 + f_6 + f_8 ) \\
f_6 &= f_8 \\
\end{split}
\end{equation}

\begin{equation}
\begin{split}
f_5 &= f_7 = \frac{1}{2}(\rho_{out} - (f_0 + f_1 + f_2 + f_3 +f_4 + f_6 + f_8 ) )\\
f_6 &= f_8 \\
\end{split}
\end{equation}

\subsubsection{Outlet top}

$\rho_{out}$ is specified and known.
$f_1$, $f_2$ and $f_5$ are known after streaming from adjacent nodes.
In the no-slip case, $u_x=u_y=0$.

The unknowns are $f_3$, $f_4$ $f_6$, $f_7$ and $f_8$.

Using NEBB at the inlet and taking into account that $u_x$ and $u_y$ are $0$:
\begin{equation}
\begin{split}
f_3 &= f_1+(f_3^{eq}-f_1^{eq}) = f_1 \\
f_4 &= f_2+(f_2^{eq}-f_4^{eq}) = f_2
\end{split}
\end{equation}

Using \eqref{Inlet_rho}
\begin{equation}
\begin{split}
f_6 + f_8 &= \rho_{out} - (f_0 + f_1 + f_2 + f_3 + f_4 + f_5 + f_7 ) \\
f_6 &= f_8 \\
f_7 &= f_5 \\
\end{split}
\end{equation}

\begin{equation}
\begin{split}
f_6 + f_8 &= 2 f_6 = 2 f_8 = \rho_{out} - (f_0 + f_1 + f_2 + f_3 + f_4 + f_5 + f_7 ) \\
f_6 &= f_8 \\
f_7 &= f_5 \\
\end{split}
\end{equation}

\begin{equation}
\begin{split}
f_6 &= f_8 = \frac{1}{2}(\rho_{out} - (f_0 + f_1 + f_2 + f_3 +f_4 + f_5 + f_7 ) )\\
f_7 &= f_5 \\
\end{split}
\end{equation}

\section{Open boundary - NRBC-CBC}
\subsection{Consistency}
The CBC methods require to set both the density and the velocity to ensure that the transverse acoustic waves are not reflected. 
If another type of BC is set for the adjacent top and bottom boundaries, this may create a conflict at the corners. 
Numerical experiments show that this leads to instabilities and that the configuration is impractical.
For this reason, absorbing layers should be preferred.

\subsection{CBC - TK method 12.4.2}
At the outlet cells, the moments $\rho$, $u_x$ and $u_y$ are calculated using the NRBC condition and are therefore known.
TK §12.4.2 page 525: "The simplest choice is to replace the distribution functions at the boundary with an equilibrium
distribution determined by the macroscopic variables $\rho$, $u_x$ and $u_y$."


After the collision and streaming steps, the distributions $f_1$, $f_2$, $f_4$, $f_5$, $f_8$ are known.
The missing distributions are $f_3$, $f_6$ and $f_7$ and their expression have been derived in the section above for prescribed velocity at the outlet:
\begin{equation}
\begin{split}
f_3 &= f_1 - \frac{2}{3} \rho u_x \\
f_6 &= f_8 - \frac{1}{2}( f_2 - f_4) - \frac{1}{6} \rho u_x  + \frac{1}{2}\rho u_y\\
f_7 &= f_5 + \frac{1}{2} (f_2 - f_4) - \frac{1}{6} \rho u_x  - \frac{1}{2}\rho u_y\\
\end{split}
\end{equation}
The above modifications ensure that the velocity at the outlet is equal to the specified velocity.
According to \cite{GW} $f_0$ should be adjusted to ensure mass conservation.

After modification, the density is
$$\rho^{\prime} = f_0 + f_1 + f_2 + f_3 +f_4 + f_5 + f_6 + f_7 + f_8 $$
Let $f^{\prime}_0$ be the modified central distrib required to ensure $\rho = \rho_{out}$:
$$\rho_{out} = f^{\prime}_0 + f_1 + f_2 + f_3 +f_4 + f_5 + f_6 + f_7 + f_8 $$
So that:
$$f^{\prime}_0 = f_0 + \rho_{out} - \rho^{\prime}$$ 



\subsection{Outlet - GW method}
Use the BC2, BC3 or BC4 methods defined in \cite{LC}.

\begin{enumerate}
    \item BC2 = ZH = NEBB: compute missing distribs using the specified $\rho_b$ and $u_x$, then adjust only $f_0$ so ensure that the moments $\rho_b$ and $u_x$ are recovered when computing moments
    \item BC3 = Regularized: compute missing distribs using the specified $\rho_b$ and $u_x$
    \item BC4 = Regularized FD: compute missing distribs using the specified $\rho_b$ and $u_x$
\end{enumerate}

\newpage
\begin{thebibliography}{9}

\bibitem{TK} 
Timm Krüger et al. 
\textit{The Lattice Boltzmann Method}, Springer, 2017.
 
\bibitem{NP} 
N. Pellerin, S. Leclaire and M. Reggio. 
\textit{An implementation of the Spalart–Allmaras turbulence
model in a multi-domain lattice Boltzmann method for
solving turbulent airfoil flows}, Computers and Mathematics with Applications 70 (2015) 3001–3018.
 
\bibitem{ZH} 
Qisu Zou  and Xiaoyi He.
\textit{On pressure and velocity flow boundary conditions and bounceback for the lattice Boltzmann BGK model}, arXiv:comp-gas/9611001v1, Nov. 1996.
 
\bibitem{LC} 
J. Latt and B. Chopard.
\textit{Straight velocity boundaries in the lattice Boltzmann method}, April 2008.

\bibitem{MS}
Martin B. Schlaffer.
\textit{Non-reflecting Boundary Conditions for the Lattice Boltzmann Method}, Dissertation 28.05.2013.
 
\bibitem{DH}
D. Heube, A. Bartel, M. Ehrhardt.
\textit{Characteristic boundary conditions in the lattice Boltzmann method for fluid and gas dynamics}, Journal of Computational and Applied Mathematics, May 2014.

 
\bibitem{GW}
G. Wissocq, N. Gourdain, O. Malaspinas, A. Eyssartier.
\textit{Regularized characteristic boundary conditions for the Lattice-Boltzmann methods at high Reynolds number flows},  (2017) Journal of Computational Physics, vol. 331. pp. 1-18. ISSN 0021-9991.
 

\end{thebibliography}

\end{document}





